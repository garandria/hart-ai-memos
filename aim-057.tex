\documentclass{tufte-handout}

\usepackage{graphicx} % allow embedded images
  \setkeys{Gin}{width=\linewidth,totalheight=\textheight,keepaspectratio}
  \graphicspath{{graphics/}} % set of paths to search for images
\usepackage{amsmath}  % extended mathematics
\usepackage{fancyvrb} % extended verbatim environments
  \fvset{fontsize=\normalsize}% default font size for fancy-verbatim environments
\usepackage{minted}
\usemintedstyle{xcode}
\usepackage{listings}
\usepackage{enumitem}
\usepackage{tikz}
% Standardize command font styles and environments
\newcommand{\doccmd}[1]{\texttt{\textbackslash#1}}% command name -- adds backslash automatically
\newcommand{\docopt}[1]{\ensuremath{\langle}\textrm{\textit{#1}}\ensuremath{\rangle}}% optional command argument
\newcommand{\docarg}[1]{\textrm{\textit{#1}}}% (required) command argument
\newcommand{\docenv}[1]{\textsf{#1}}% environment name
\newcommand{\docpkg}[1]{\texttt{#1}}% package name
\newcommand{\doccls}[1]{\texttt{#1}}% document class name
\newcommand{\docclsopt}[1]{\texttt{#1}}% document class option name
\newenvironment{docspec}{\begin{quote}\noindent}{\end{quote}}% command specification environment


\lstnewenvironment{sexp}{\lstset{basicstyle=\ttfamily,language=Lisp}}{}
\lstnewenvironment{mexp}{\lstset{basicstyle=\ttfamily,mathescape}}{}
\newcommand{\macroequiv}[0]{\overset{\mathtt{MACRO}}{\mathbf{\equiv}}}

\title{MACRO Definition for LISP}
\author[Timothy P. Hart]{Timothy P. Hart}
\date{October 12, 1963} % without \date command, current date is supplied


\begin{document}

\maketitle% this prints the handout title, author, and date

In LISP 1.5 special forms are used for three logically separate purposes: a)~to
reach the alist, b)~to allow function to have an indefinite number of arguments,
and c)~to keep arguments from being evaluated.

New LISP interpreters can easily satisfy need (a)~by making the alist n
SPECIAL-type or APVAL-type entity. Uses (b) and (c) can be replaced by
incorporating a \texttt{MACRO} instruction expander in \texttt{define}. I am
proposing such expander.

\begin{enumerate}
\item The property list of a macro instruction will have the indicator
  \texttt{MACRO} followed by a function of one argument, a form beginning with
  the macro's name and whose value will replace the original form in all
  function definitions.
\item The function \texttt{macro[l]} will define macro's just as
  \texttt{define[l]} defines functions.
\item \texttt{define} will be modified to make macro expansions.
\end{enumerate}

Examples:

\begin{enumerate}
\item The existing FEXPR \texttt{csetq} may be replaced by the macro definition:
  \begin{sexp}
    MACRO ((
    (STASH (LAMBDA (FORM) (LIST (QUOTE SETQ) (CADAR FORM) (LIST (CONS (CADR FORM)
    (CADAR FORM))) )))
    ))
  \end{sexp}

\item A new macro \texttt{stash} will generate the form found frequently in
  \texttt{PROG'a}:

  \begin{mexp}
    x := cons[form;x]
  \end{mexp}
  
  using the macro \texttt{stash}, one might write instead of the above:
  % \begin{sexp}
  %   (STASH FORM X).
  % \end{sexp}

  \texttt{Stash} may be defined by:
  \begin{sexp}
    MACRO ((
    (STASH (LAMBDA (FORM) (LIST (QUOTE SETQ) (CADAR FORM) (LIST (CONS (CADR FORM)
    (CADAR FORM))) )))
    ))
  \end{sexp}
\item New macros may be defined in terms of old. \texttt{enter} is a macro for
  adding a new entry to the table (dotted pairs) stored as the value of a
  program variable.

  \begin{mexp}
    enter[form] $\macroequiv{}$ list[STASH;list[CONS;cadr[form];caddr[form]];
    cadddr[form]]
  \end{mexp}
  \paragraph{}%
  Incidentally, use of macros will alleviate the present difficulty resulting
  from the 90 LISP compiler's only knowing about those fexprs in existence at
  its birth.
  \paragraph{}%
  The macro defining function \texttt{macro[l]} is easily defined:
  
  \begin{mexp}
    macro[l] $\equiv$ deflist[l;MACRO]
  \end{mexp}

  \paragraph{}%
  The new \texttt{define} is a little harder:

  \begin{mexp}
    define[l] $\equiv$ deflist[mdef[l];EXPR]
    mdef[l] $\equiv$ [
            atom [l] $\rightarrow$ l;
            eg[car[l];QUOTE] $\rightarrow$ l;
            member[car[l];(LAMBDA LABEL PROG)] $\rightarrow$
                cons[car[l];cons[cadr[l];mdef[caddr[l]]]]
            get[car[l];MACRO] $\rightarrow$ mdef[get[car[l]; MACRO]
        [l]]
            T $\rightarrow$ maplist[l;$\lambda$[[j];mdef[car[j]]]]]
  \end{mexp}
\item The macro for \texttt{select} illustrates the use of macros as a means of
  allowing functions of an arbitrary number of arguments:
  \begin{mexp}
    select[form] $\macroequiv{}$ $\lambda$[[g];
        list[list[LAMBDA;list[g];cons[COND;
            maplist[cadr[form];$\lambda$[[l];
                [null[cdr[l]] $\rightarrow$ list[T;car[l]];
                    T $\rightarrow$ [list[EQ;g;caar[l]];cadar[l]]]]]
            ]];cadr[form]]][gensym[]]
  \end{mexp}
  
\end{enumerate}

\end{document}

%%% Local Variables:
%%% mode: latex
%%% TeX-command-extra-options: "-shell-escape"
%%% TeX-master: "aim-057"
%%% End:
